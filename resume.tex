% LaTeX file for resume 
% This file uses the resume document class (res.cls)

\documentclass[line,overlapped]{res} 
% the margin option causes section titles to appear to the left of body text 
%\addtolength{\hoffset}{-1in}
%\hoffset=-1in
\textwidth=6.6in % increase textwidth to get smaller right margin
\resumewidth=7.0in % increase textwidth to get smaller right margin
%\sectionwidth=5.0in
\sectionskip=0.1in
\usepackage{helvetica} % uses helvetica postscript font (download helvetica.sty)
%\usepackage{newcent}   % uses new century schoolbook postscript font 

% moves doc over to the left
\setlength{\oddsidemargin}{-.25in}
\setlength{\evensidemargin}{-.25in}


\renewcommand{\thefootnote}{\fnsymbol{footnote}}

\begin{document}

\name{Jordan A. Lewis} % the \\[12pt] adds a blank line after name

\address{jordanthelewis@gmail.com \hspace{1.63in} 325 E. 16th St. Apt. 2, Brooklyn, NY 11226}
%\address{(917) 974-7144 \hspace{2.81in} Permanent: 6 Warren Place, Brooklyn, NY 11201}


\newsectionwidth{.2in}
%\sectionfont{\sl}
\begin{resume}

\section{Education}
{\sl Bachelor of Science,} Computer Science, The University of Chicago \hfill 2011\\
%{\sl Associate of Arts,} Bard College\hfill 2007
% Probably don't need high school diploma along with the above on here
%Bard High School Early College \hfill 2008

\section{Work Experience}
{\bf Senior Software Engineer,} Knewton \hfill Sep. 2011-Present\\
{\bf Intern Systems Engineer,} RethinkDB \hfill December 2010\\
{\bf Homework/Lab grader,} CMSC 15400/15100, The University of Chicago \hfill Spring, Autumn 2010\\
{\bf Intern Software Developer,} The Manticore Project, The University of Chicago \hfill Summer 2010
\begin{itemize} \itemsep -2pt
    \item Continued development on and maintained an Objective-C++/Cocoa log file visualization program for Manticore, a functional parallel programming language and compiler
\end{itemize}

\vspace{-10pt}

{\bf Sys. Admin.,} Computation Institute, The University of Chicago \hfill Aug. 2008-Sep. 2009
\begin{itemize} \itemsep -2pt %reduce space between items
    \item Maintained 2 clusters of around 200 Linux servers each across 2 physical sites; helped develop a software suite to ease cluster error recovery and management
    %\item Assembled and cabled new compute nodes and storage servers; diagnosed, maintained and replaced failed hardware components
    %\item Administered multiple Cyclades console servers and APC PDUs and UPSs
\end{itemize} \itemsep -2pt
\vspace{-10pt}
{\bf Software Developer,} CSPP, The University of Chicago \hfill Jun. 2009-Aug. 2009
\begin{itemize} \itemsep -2pt
    \item Designed and implemented a SQLite database to track information relating to incoming students, incorporating multiple disparate and incompatible sources of data into a unified schema
    %\item Worked with staff members to discover the best workflow for the particulars of this application, and created a simple web interface to match that workflow.
\end{itemize}
\vspace{-10pt}

{\bf Software Developer,} Econnectix, Chicago, IL\hfill Jan. 2008-Apr. 2009
\begin{itemize} \itemsep -2pt % reduce space between items
    \item Designed and implemented a system health tracking and management program to detect and handle dangerous physical conditions for a high-availability embedded device
    \item Designed and implemented ``support tunnel'' instant tech-support system from scratch in three weeks, allowing customers to get help directly from a device's user interface
    \item Collaborated on a major refactoring of a storage volume management and server state configuration system for a fibre channel SCSI storage device
\end{itemize}
\vspace{-10pt}

{\bf Software Developer,} Vim, Google Summer of Code, \hfill Summer 2008-Autumn 2010
\begin{itemize} \itemsep -2pt
    \item Designed and implemented undo tree persistence, one of the Vim community's most requested feature additions (undos/redos automatically saved upon closing and restored upon reopening a file)\footnotemark[2]
    \item Continued to support the feature in spite of difficulties getting it pushed upstream, continued development via a separate channel until eventual upstream inclusion
\end{itemize}

%%{\bf Tech Support,} Lower East Side Girls Club, NY, NY \hfill Feb. 2006-Sep. 2007
%%\begin{itemize} \itemsep -2pt
%%    \item Developed infrastructure for automated backups of computers in a small network
%%    \item Simplified and improved the organization's ethernet network topology
%%    \item Maintained and upgraded hardware and software
%%\end{itemize}

%%{\bf Intern,} National Park Service Youth Conservation Corps, NY, NY \hfill Jun.-Sep. 2007
%%\begin{itemize} \itemsep -2pt
%%    \item Collaborated with Governor's Island managerial and grounds staff on physical maintenance and development tasks
%%    \item Presented information and historical background about the National Parks Service and Governor's Island to other young adults
%%\end{itemize}
%%{\bf Counselor and Trip Leader,} Tanager Lodge, Merrill, NY \hfill Jun.-Aug. 2006
%%\begin{itemize} \itemsep -2pt
%%    \item Responsible for safety and well-being of four 8-10-year-olds during a 7 week wilderness summer program
%%    \item Co-led multi-day wilderness backpacking trips for groups of 8-14-year-olds
%%\end{itemize}
\section{Academic Projects}
%{\bf Optimizing compiler for an SML-like language} The Univ. of Chicago, CMSC 22610/20 \hfill Winter-Spring 2009\\
{\bf Tensor Rundown, a multiplayer 3D racing game}\footnotemark[2] The Univ. of Chicago, CMSC 23800 \hfill Spring 2010\\
{\bf Prototype SML-like Module System}\footnotemark[2] The Univ. of Chicago, CMSC 33600 \hfill Winter 2010\\
{\bf Simple MIPS Simulator}\footnotemark[2] The University of Chicago, CMSC 22200 \hfill Autumn 2009\\
{\bf Simple RDBMS,}\footnotemark[2] The University of Chicago, CMSC 23500 \hfill Spring 2009
\begin{itemize} \itemsep -2pt
    \item Collaborated with the class to build a simple RDBMS in C from the ground up, including a B-Tree backend, a database virtual machine, a SQL-to-VM code generator, and a simple shell to interact with the system
\end{itemize}
\vspace{-10pt}
{\bf TCP-like implementation; IP router}\footnotemark[2] The University of Chicago, CMSC 23300 \hfill Autumn 2008
\begin{itemize} \itemsep -2pt
    \item In a two-person team, implemented a TCP-like reliable transport protocol called STCP on top of a simulated unreliable network layer, and an IP router with proper support for ARP, ICMP, and routing directly over Ethernet packets
\end{itemize}


% Tabulate Computer Skills; p{3in} defines paragraph 3 inches wide
\section{Skills}
   \begin{tabular}{l p{4in}}
   {\sl Languages:}&/(Objective-)?C(++)?/, SML, Python, Scheme, Bash, GLSL\\
   {\sl Graphical Toolkits:}&Cocoa, OpenGL\\
   {\sl Tools: } & Vim, gdb, CVS, Subversion, git, SQLite, lex, yacc\\
   {\sl OS:} & Linux (Arch, Debian, Gentoo, Scientific, Ubuntu), OS X\\
 \end{tabular}

\end{resume}
\footnotetext[2]{Source code available at http://github.com/jordanlewis/, or upon request}
\end{document}
